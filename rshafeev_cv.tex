%flagCMYK
\documentclass[helvetica,openbib,nologo,notitle,totpages]{europecv}
\usepackage[T1]{fontenc}
\usepackage{graphicx}
\usepackage[a4paper,top=1cm,left=1cm,right=1cm,bottom=2cm]{geometry}
\usepackage[english]{babel}
\usepackage{url}
\usepackage{hyperref}


\ecvname{Shafeev, Roman}
\ecvfootername{Shafeev Roman}
\ecvaddress{18/2 f.345 European avenue, Kudrovo, Leningrad region, Russia}
\ecvtelephone[+7 965 078 43 30]{Skype ID: roma.shafeev }
\ecvemail{\url{r.a.shafeev@yandex.com}}


\ecvnationality{Russian}
\ecvdateofbirth{Feb 14 1990}
\ecvgender{male}
\ecvpicture[width=4cm]{photo2.jpg}
\ecvfootnote{For more information call me}

\begin{document}
\selectlanguage{english}


\begin{europecv}
\ecvpersonalinfo[5pt]

\ecvsection{Work experience}

\ecvitem{Date}{March 2018 --July 2019}
\ecvitem{Occupation or position held}{Product Lead ( Connected Vehicle Cloud ), ARRIVAL LTD}


\ecvitem{Date}{June 2017 -- March 2018}
\ecvitem{Occupation or position held}{Product Lead ( Insurance Telematics ), StarLine}

% ========== Starline Developing ========== %
\ecvitem{}{}
\ecvitem{\textbf{Date}}{\textbf{July 2015 -- June 2017 , 2 years}} 
\ecvitem{\textbf{Company}}{\textbf{StarLine}}
\ecvitem{\textbf{Position Held}}{\textbf{C++/Python Backend Developer}}
\ecvitem{Roles}{	
$\bullet$ C++ Backend Developing

$\bullet$ Python API Backend Developing

$\bullet$ Technical Design

$\bullet$ Applied Mathematics Research

}

\ecvitem{Key Achievements}{
	$\bullet$ designed and implemented  MSA with Decomposition way for Telemetry Monoliths Server using RabbitMQ as Services Messages Bus.
	
	$\bullet$ designed and implemented remote communication between client applications(mobile apps, web portal) and connected vehicles through $client \leftrightarrow api \leftrightarrow platform \leftrightarrow vehicle$ on Connected Vehicle Platform side using redis pub/sub channels.
	
	$\bullet$ developed backend micro-services with horizontal scaling to receive and to collect vehicle`s telemetry of more than 300K connected vehicles.

	$\bullet$ implemented 'insurance-telematics' feature to calculate insurance score based on raw data of vehicle accelerometer which collected from the vehicle on cloud side.
	
	$\bullet$ designed and implemented heuristic algorithm to solve 'anti-star' problem to filter noise and broken track points in real-time.
}

\ecvitem{Technical Skills}{
$\bullet$ Connected Vehicle Development: c++11, cmake, poco, stl, libevent, protobuf

$\bullet$ API Development: python3, tornado, sqlalchemy

$\bullet$ Services Communication Bus: rabbitmq (via SimpleAmqpClient for c++ / via pika for python)

$\bullet$ Databases:  redis (via hiredis), mysql (via mysqlcppconn), oracle (via occi)
 
}

% ========== KhPI Developing ========== %
\ecvitem{}{}
\ecvitem{}{}
\ecvitem{\textbf{Date}}{\textbf{September 2013 -- June 2015 , 1 year 10 months}} 
\ecvitem{\textbf{Company}}{\textbf{NTU "KhPI", Ukraine}}
\ecvitem{\textbf{Position Held}}{\textbf{Researcher of the Department of Computer Mathematics and Mathematical Modeling}}
\ecvitem{Roles}{
		$\bullet$ Applied Mathematics Research
		
		$\bullet$ Conducting laboratory and practical classes of C++ programming
}
\ecvitem{Key Achievements}{
		$\bullet$ designed and implemented matlab application which allows to find the best productive supply for each transformer with minimal losses on the  power transformers in Dushanbe (Tajikistan).
		
		$\bullet$  developed a vehicle routing java framework that uses specialized metaheuristic algorithms to calculate an optimal solution of the different classes of the static and dynamic vehicle routing problems.
}
\ecvitem{Technical Skills}{
	$\bullet$ Language: C++, Java
	
	$\bullet$ Math Tools: Matlab
}


% ========== TUHH ========== %
\ecvitem{}{}
\ecvitem{}{}
\ecvitem{\textbf{Date}}{\textbf{July 2011 -- November 2011 , 5 months}} 
\ecvitem{\textbf{Company}}{\textbf{Hamburg University of Technology-TUHH}}
\ecvitem{\textbf{Position Held}}{\textbf{Software Developer}}
\ecvitem{Roles}{
	$\bullet$ Applied Mathematics Research
	
	$\bullet$ Conducting laboratory and practical classes of C++ programming
}
\ecvitem{Key Achievements}{
	$\bullet$ built model which follows a rigorous development process framework, where model	validity is ensured by using Supply Chain Operations Reference as theoretical process	framework using Anylogic Modeling Platform. An agent based simulation platform is presented for generic supply chain modeling adding flexibility and configurability over existing models.
	
	$\bullet$ developed UI-tool which allows to design delivery routes between the points of the delivery chains and export them  in .accdb format to have the ability to use them in Anylogic into supply chain model.
}

\ecvitem{Technical Skills}{
	
	$\bullet$ Modeling Tools: Anylogic 6.6 
	
	$\bullet$ Development: C++, WinAPI/MFC, Visual Studio 2008, OpenStreetMap
}



\ecvitem{}{}
\ecvitem{}{}
\ecvsection{Personal skills and~competences}

\ecvmothertongue[5pt]{Russian}
\ecvitem{\large Other language(s)}{English}
\ecvlanguageheader{(*)}
\ecvlanguage{English}{\ecvBOne}{\ecvCOne}{\ecvBTwo }{\ecvBTwo }{\ecvBTwo }
\ecvlanguagefooter[10pt]{(*)}


\ecvitem{\large Actual Technical skills and competences}{}
\ecvitem{ }{$\bullet$ \textbf{Operating System Experiences}}
\ecvitem{ }{ Linux (debian, ubuntu, centos)}

\ecvitem{ }{$\bullet$ \textbf{Programming Languages}}
\ecvitem{ }{ C++ 11+, Python 3, SQL,  PL/pgSQL, Java / Scala(basic level)}

\ecvitem{ }{$\bullet$ \textbf{Database Management Systems}}
\ecvitem{ }{ PostgreSQL, Oracle, Redis, InfluxDB, Cassandra, ClickHouse (basic level)}


\ecvitem{ }{$\bullet$ \textbf{Python technologies \& frameworks}}
\ecvitem{ }{ Tornado, aiohttp, sqlalchemy, WAMP}

\ecvitem{ }{$\bullet$ \textbf{C++ technologies \& frameworks}}
\ecvitem{ }{ cmake, STL, libevent, nghhtp2}


\ecvitem{ }{$\bullet$ \textbf{Data Streaming \& Services Bus}}
\ecvitem{ }{ RabbitMQ, Apache Kafka, Temporal.io }

\ecvitem{ }{$\bullet$ \textbf{Orchestration \& Containerization}}
\ecvitem{ }{ k8s, docker compose, docker swarm }

\ecvitem{ }{$\bullet$ \textbf{Cloud Computing Services}}
\ecvitem{ }{ GCP k8s, Google Pub/Sub}

\ecvitem{ }{$\bullet$ \textbf{Infra \& CI tools}}
\ecvitem{ }{ Jenkins CI, Gitlab CI, Ansible, Terraform}

\ecvitem{ }{$\bullet$ \textbf{Development tools}}
\ecvitem{ }{ PyCharm, CLion, Intellij Idea}

\ecvitem{ }{$\bullet$ \textbf{Version Control Systems}}
\ecvitem{ }{ Git}

\ecvitem{ }{$\bullet$ \textbf{Other skills}}
\ecvitem{ }{ Mathematics: MatLab, R Studio}
\ecvitem{ }{ Simulation: Rational Rose, Anylogic}

\ecvitem[10pt]{ }{ }


\ecvitem{SOURCE CODE}{}
\ecvitem{ }{\textbf{Source code, demonstration video and documentation of my projects: } }
\ecvitem{ }{\url{https://github.com/rshafeev}}

\ecvsection{Education and training}

\ecvitem{Place and Date}{National Technical University "Kharkov Polytechnic Institute", Ukraine, 2013 -- 2016}
\ecvitem{Specialty}{Mathematical modeling and computational methods}
\ecvitem{Title of qualification awarded}{passed PhD minimum, successful completion of postgraduate study}
\ecvitem[15pt]{Thesis theme}{Development of mathematical models and methods to solve the Dynamic Vehicle Routing Problem with uncertain input parameters}

\ecvitem{Place and Date}{National Technical University "Kharkov Polytechnic Institute", Ukraine}
\ecvitem{ }{Computer Mathematics and Mathematical Modeling department, 2011 -- 2013}
\ecvitem[15pt]{Title of qualification awarded}{Master`s degree in Applied Mathematics with excellence}

\ecvitem{Place and Date}{National Technical University "Kharkov Polytechnic Institute", Ukraine}
\ecvitem{ }{Computer Mathematics and Mathematical Modeling department, 2007 -- 2011}
\ecvitem{Title of qualification awarded}{Bachelor`s degree in Applied Mathematics with excellence}
\ecvitem{Principal subjects covered}{Mathematical Analysis}
\ecvitem{ }{Discrete Mathematics}
\ecvitem{ }{Programming (C,C++)}
\ecvitem{ }{Probability Theory and Mathematical Statistics}
\ecvitem{ }{Object Oriented Programming}
\ecvitem{ }{Numerical Methods}
\ecvitem{ }{Optimization Methods}
\ecvitem{ }{Logical Algorithms and Artificial Intelligence Systems}
\ecvitem{ }{Control Theory}
\ecvitem{ }{Development of Information Systems (Java, IDEF, Web 2.0)}
\ecvitem{ }{Computer Simulation}
\ecvitem{ }{Distributed Information Systems(Oracle)}
\ecvitem{ }{Actuarial Mathematics}



\ecvitem{PUBLICATIONS}{
	\begin{list}{\textbullet}{}
	  \item R. Shafeev.  Investigation of tuning parameters of Tabu Search algorithm and its modification  for solving the static Routing Courier Delivery Problem.  Kharkov NTU "KhPI" , 2016, 18 p.
	  
	 \item Lyubchyk L.M., Kolbasin V.A., Shafeev R.A. Nonlinear Signal Reconstruction based on Recursive Moving Window Kernel Method. / IDAACS, Warsaw, Poland, 2015, 6 p.
	  	  
	  \item R. Shafeev. A new metaheuristic algorithm for Solving the Transportation Problem with Time Constraints / L. Lyubchik // Vestnik NTU "KhPI". -- Kharkov: NTU "KhPI", 2013.  -- No3 (977). -- p. 35--39.   
	  
	 \item Shafeev R.A. A Development of SaaS service to solve dynamic vehicle routing problem / System analysis and information technologies: SAIT, Kyiv, 2013.
	  	  
	  \item R. Shafeev. Relationship between the Vehicle Routing Problem with Time Windows and the Assignment Problem.  // Theoretical and Applied Aspects of Cybernetics.  -- Kiev: Bukrek, 2012. -- p.145--149.
	  
	\end{list}
}
\ecvitem[5pt]{ }{ }
		
\ecvitem{SCIENTIFIC WORK}{
\begin{list}{\textbullet}{}

	\item May 2013, I presented the research work, devoted of development of client-server information system for solving the Dynamic Vehicle Routing Problem at the XV International Conference on Science and Technology "System	Analysis and Information Technologies" at the National Technical University “KPI”, Kiev, Ukraine.
	
	\item March 2012, The winner (1`st place) of the all-Ukrainian competition of the research student works, section "Informatics and Cybernetics", Vinnytsia, Ukraine.
	
	\item September 2011, participant of the International Conference of Logistics at the Hamburg	University of Technology, Hamburg, Germany.
	
	\item October 2010, I presented the research work, devoted to effects of electromagnetic fields on the complex biological objects at the Vth International conference “Environmental aspects of the technological security of the regions“ at the National Automobile and Road University, Kharkov, Ukraine.
	
	\item May 2010, I presented the research work, devoted to numerical simulation of the motion of celestial bodies at the XII International Conference on Science and Technology “System	Analysis and Information Technologies“ at the National Technical University "KPI", Kiev, Ukraine.
	
	\item May 2007, The winner (2nd place) of the third stage of the all-Ukrainian competition of research carried out by the students-members of the Ukrainian Small Academy of Sciences, section "Computer networks, databases and data banks", Kiev, Ukraine.
	
	\item December 2006, The winner (1nd place) of the second stage of the all-Ukrainian competition of research carried out by the students-members of the Ukrainian Small Academy of Sciences, section "Computer networks, databases and data banks", Zaporozhye, Ukraine.	
	
\end{list}
}



\ecvsection{Additional information}
\ecvitem{GRANTS}{Grant of Government of Ukraine, 2010--2011.}
\ecvitem[10pt]{}{Grant of the “DAAD-East European Partnership Exchange” funding framework between “National	Technical University” (Kharkov, Ukraine) and “Hamburg University of Technology-TUHH” (Germany). During	the internship, I worked as a team member, which developed Supply Chain Management project, Hamburg (Germany), July -- October 2011.}


\end{europecv}


\end{document} 