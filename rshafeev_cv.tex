%flagCMYK
\documentclass[helvetica,openbib,nologo,notitle,totpages]{europecv}
\usepackage[T1]{fontenc}
\usepackage{graphicx}
\usepackage[a4paper,top=1cm,left=1cm,right=1cm,bottom=2cm]{geometry}
\usepackage[english]{babel}
\usepackage{url}
\usepackage{hyperref}
\usepackage{xcolor}

\makeatletter
\newcommand*\bigcdot{\mathpalette\bigcdot@{.5}}
\newcommand*\bigcdot@[2]{\mathbin{\vcenter{\hbox{\scalebox{#2}{$\m@th#1\bullet$}}}}}
\makeatother

\ecvname{Shafeev, Roman}
\ecvfootername{Shafeev Roman}
\ecvaddress{Saint-Petersburg, Russia}

\ecvemail{\url{r.a.shafeev@yandex.com}, telegram:  @rshafeev}


\ecvnationality{Russian}
\ecvdateofbirth{Feb 14 1990}
\ecvgender{male}
\ecvpicture[width=4cm]{photo3.jpg}
\ecvfootnote{For more information call me}

\begin{document}
\selectlanguage{english}


\begin{europecv}
\ecvpersonalinfo[5pt]

\ecvsection{Work experience}

% ========== BUREAU 1440 ========= %

\ecvitem{}{}
\ecvitem{\textbf{Date}}{\textbf{January 2024 -- December 2024  \textcolor{gray}{ 1 year} }} 
\ecvitem{\textbf{Company}}{\textbf{BUREAU 1440}}
\ecvitem{\textbf{Position Held}}{\textbf{ Head of Architectural Design Department }}
\ecvitem{Roles}{
    $\bigcdot$ Leading a group of architects and system analysts.
    
	$\bigcdot$ Architecture Design of Satellite Internet Constellation Network Management System.
	
	$\bigcdot$ Managing of architectural documentation.
	
	$\bigcdot$ Defining, collecting, refining and organizing product functionality requirements.
	
}

\ecvitem{Key Achievements}{
	$\bigcdot$ implemented SDLC for architects and system analysts.
	
	$\bigcdot$ implemented the Architecture as Code approach and organized the development process using ADRs.
	
	$\bigcdot$  made an HLD for a Unified Space Communications Network Management System: the system allows to remote configuration, to execute control actions (for example, impulses for satellite maneuvering), to update satellite software, to collect telemetry, metrics, logs, and to detect faults of the network element or satellite.
	
	$\bigcdot$ designed a new Earth-Satellite communication protocol.
	
	$\bigcdot$ designed a workflow orchestration subsystem that helps operational operators coordinate work (configuration, software updates, maneuvering, etc.) for their mutual consistency in a semi-automatic mode.
	
	$\bigcdot$ designed a solution for automating the process of refining and updating the forecast of the spacecraft's trajectory. 
	
	$\bigcdot$ organized the migration of the management system to k8s.
		
}

\ecvitem{Tech stack}{
	$\bigcdot$ Programming Language: java, c++ , typescript (frontend)
		
	$\bigcdot$ Bus: Apache Kafka, Temporal.io
	
	$\bigcdot$ Databases: Cassandra, ClickHouse, Postgresql
	
	$\bigcdot$ NMS: netconf, YANG, sysrepo, GNMI
	
	$\bigcdot$ Protocols: QUIC, grpc, CANBus
	
}

% ========== MTS AI ========= %

\ecvitem{}{}
\ecvitem{\textbf{Date}}{\textbf{July 2022 -- December 2023  \textcolor{gray}{ 1 year 6 months } }} 
\ecvitem{\textbf{Company}}{\textbf{MTS AI}}
\ecvitem{\textbf{Position Held}}{\textbf{Chief Solutions Architect }}
\ecvitem{Roles}{	
	$\bigcdot$Architecture Design of the \href{https://mts.ai/ru/product/audiogram/}{Audiogram product} for speech synthesis and recognition (TTS/ASR).	
	
	$\bigcdot$ Managing of architectural documentation.
	
	$\bigcdot$ Defining, collecting, refining and organizing product functionality requirements.
}

\ecvitem{Key Achievements}{
	
	$\bigcdot$ designed Identity Access Management with an authorization model based on json-policies.

	$\bigcdot$ designed a solution for horizontal scaling of Audiogram product services; organized the migration of the product to k8s.

	$\bigcdot$ completed the transition from SOA to MSA model.

	$\bigcdot$ implemented such engineering practices as 19-apps factors, observability, structured logging, distributed tracing.
			
	$\bigcdot$ designed a solution for identifying and verifying a speaker by his voice.
	
	$\bigcdot$ designed a solution for determining the emotions and the age group of a speaker during speech recognition from an audio stream.
	
	$\bigcdot$ designed an Audio Archive for recording streaming audio.
	
	$\bigcdot$ designed a solution for collecting and processing dialogues of communication between chatbots and users in the NLP Analytics system.
		
}

\ecvitem{Tech stack}{
	$\bigcdot$ Programming Language: python, go
	
	$\bigcdot$ Bus: Apache Kafka
	
	$\bigcdot$ Databases: Postgresql, MongoDB, OpenSearch, Redis, Milvus, etcd
	
	$\bigcdot$ Machine Learning: Triton Server
	
}

% ========== RingCentral  ========== %

\ecvitem{}{}
\ecvitem{\textbf{Date}}{\textbf{Sep 2019 -- January 2022 \textcolor{gray}{ / 2 years 5 months} }} 
\ecvitem{\textbf{Company}}{\textbf{RingCentral}}
\ecvitem{\textbf{Position Held}}{\textbf{TeamLead -> Solutions Architect }}
\ecvitem{Roles}{		
	$\bigcdot$ Architecture Design of the RingCentral Analytics Product
	
	$\bigcdot$ Managing of architectural documentation.
	
	$\bigcdot$ Defining, collecting, refining and organizing product functionality requirements.
	
	$\bigcdot$ backend development of IP telephony services in C++.
}

\ecvitem{Key Achievements [Architect]}{
	$\bigcdot$ designed BI-analytics system for video-conferences and webinars.

	$\bigcdot$ designed technical solution to async generate and download Analytics Reports.
		
	$\bigcdot$ designed technical solution to build `Webinar Marketing` Analytics. The Analytics System collects any actions of the attendee starting from the moment of registration in the webinar and ending with the completion of his participation in the webinar to build funnel trends for Marketers.

	$\bigcdot$ designed technical solution to reorganize RingCentral Analytics System to ensure GDPR and Schrems II Compliance and built regional clusters for EU and APAC.

	$\bigcdot$ made an audit of architectural documentation and actualized system and functional specifications of RingCentral Analytics System; built py-tool which generate draw.io interactive architectural diagrams from configuration files and applications artifacts and publish them into Confluence using Confluence API and integrated him to GitLab CI. 
	
}

\ecvitem{Key Achievements [TeamLead]}{
	$\bigcdot$ built Development Process based on Self Agile methodology. 
	
	$\bigcdot$ migrated legacy Kernel Telephony Service (millions code lines) from Win to Linux Platform. 
	
	$\bigcdot$ refactored legacy Kernel Telephony Service using EDA architecture.
	
}

\ecvitem{Tech stack}{
	$\bigcdot$ Programming Language: scala, typescript (frontend), c++11, cmake, boost, stl, protobuf
	
	$\bigcdot$ Cloud Services: GCP k8s, GCP Pub/Sub, Segment	
	
	$\bigcdot$ Bus: Apache Kafka, Temporal.io
	
	$\bigcdot$ Databases: Oracle, Cassandra, ClickHouse, rocksdb, lucene
	
}


% ========== Arrival Leading ========== %

\ecvitem{}{}
\ecvitem{\textbf{Date}}{\textbf{March 2018 --July 2019 \textcolor{gray}{ / 1 year 5 months} }} 
\ecvitem{\textbf{Company}}{\textbf{ARRIVAL LTD}}
\ecvitem{\textbf{Position Held}}{\textbf{Technical Lead ( Connected Vehicle Cloud ) }}
\ecvitem{Roles}{	
	$\bigcdot$ Team Leading.
	
	$\bigcdot$ Product Roadmap Planning.

	$\bigcdot$ Architecture Design.
}

\ecvitem{Key Achievements}{
	$\bigcdot$ formed a product RnD-team.
	
	$\bigcdot$ built the Connected Vehicle Cloud for Arrival vehicles.
	
	$\bigcdot$ released a Remote Diagnostics product that allows service engineers to determine the cause of failures in an electric vehicle remotely.
	
	$\bigcdot$ released the OTA-Update product for remote updating of electric vehicle`s SW components.
	
}

\ecvitem{Tech stack}{
	$\bigcdot$ Connected Vehicle Development: c++11, cmake, nghhtp2, stl, protobuf, javascript (frontend)
	
	$\bigcdot$ API: python3, tornado, sqlalchemy, WAMP
	
	$\bigcdot$ Cloud Services: GCP k8s
	
	$\bigcdot$ Bus: RabbitMQ Cluster
	
	$\bigcdot$ Databases: PostgreSQL, InfluxDB
	
}


% ========== Starline Leading ========== %
\ecvitem{}{}
\ecvitem{\textbf{Date}}{\textbf{July 2015  -- March 2018 \textcolor{gray}{ / 2 years 9 months} }} 
\ecvitem{\textbf{Company}}{\textbf{StarLine}}
\ecvitem{\textbf{Position Held}}{\textbf{Backend Developer -> Technical Lead }}

\ecvitem{Roles}{	
	$\bigcdot$ Team Leading.
	
	$\bigcdot$ Product Roadmap Planning.
	
	$\bigcdot$ Technical Integration with Telematics Partners.
	
	$\bigcdot$ Architecture Design.
	
	$\bigcdot$ C++ and Python Backend Developing.
}

\ecvitem{Key Achievements [TechLead]}{
	$\bigcdot$ implemented 'the driving style assessment' feature to calculate driving style score.
	
	$\bigcdot$  implemented the 'insurance telematics' b2b product to calculate insurance score and crash detection/reconstruction.		
}

\ecvitem{Key Achievements [Developer]}{
	$\bigcdot$ designed b2b-integration with Insurance Partners to transfer vehicles telemetry to their systems in real-time mode.
	
	$\bigcdot$ designed and implemented  MSA Architectural concepts for Telemetry Monoliths Server Decomposition.
	
	$\bigcdot$ developed backend micro-services with horizontal scaling to receive and to collect vehicle`s telemetry of more than 300K connected vehicles.
	
}

\ecvitem{Tech stack}{
	
	$\bigcdot$ Connected Vehicle Development: c++11, cmake, poco, stl, libevent, protobuf
	
	$\bigcdot$ API Development: python3, tornado, sqlalchemy
	
	$\bigcdot$ Bus: RabbitMQ Cluster
	
	$\bigcdot$ Databases:  Redis, MySql, Oracle
}

% ========== KhPI Developing ========== %
\ecvitem{}{}
\ecvitem{}{}
\ecvitem{\textbf{Date}}{\textbf{September 2013 -- June 2015 \textcolor{gray}{ / 1 year 10 months} }} 
\ecvitem{\textbf{Company}}{\textbf{NTU "KhPI", Ukraine}}
\ecvitem{\textbf{Position Held}}{\textbf{Researcher of the Department of Computer Mathematics and Mathematical Modeling}}
\ecvitem{Roles}{
		$\bigcdot$ Applied Mathematics Research.
		
		$\bigcdot$ Conducting laboratory and practical classes of C++ programming.
}
\ecvitem{Key Achievements}{
		$\bigcdot$ designed and implemented matlab application which allows finding the best productive supply for each transformer with minimal losses on the  power transformers in Dushanbe (Tajikistan).
		
		$\bigcdot$  developed a vehicle routing java framework that uses specialized metaheuristic algorithms to calculate an optimal solution of the different classes of the static and dynamic vehicle routing problems.
}
\ecvitem{Tech stack}{
	$\bigcdot$ Language: C++, Java
	
	$\bigcdot$ Math Tools: Matlab
}


% ========== TUHH ========== %
\ecvitem{}{}
\ecvitem{}{}
\ecvitem{\textbf{Date}}{\textbf{July 2011 -- November 2011 \textcolor{gray}{ / 5 months} }} 
\ecvitem{\textbf{Company}}{\textbf{Hamburg University of Technology-TUHH}}
\ecvitem{\textbf{Position Held}}{\textbf{Software Developer}}
\ecvitem{Roles}{
	$\bigcdot$ Applied Mathematics Research.
	
	$\bigcdot$ C++ / Java Developing.
}
\ecvitem{Key Achievements}{
	$\bigcdot$ built model which follows a rigorous development process framework, where model	validity is ensured by using Supply Chain Operations Reference as theoretical process	framework using Anylogic Modeling Platform. An agent based simulation platform is presented for generic supply chain modeling adding flexibility and configurability over existing models.
	
	$\bigcdot$ developed UI-tool which allows to design delivery routes between the points of the delivery chains and export them  in .accdb format to have the ability to use them in Anylogic into supply chain model.
}

\ecvitem{Tech stack}{
	
	$\bigcdot$ Modeling Tools: Anylogic 6.6 
	
	$\bigcdot$ Development: C++, WinAPI/MFC, Visual Studio 2008, OpenStreetMap
}



\ecvitem{}{}
\ecvitem{}{}
\ecvsection{Personal skills and~competences}

\ecvmothertongue[5pt]{Russian}
\ecvitem{\large Other language(s)}{English}
\ecvlanguageheader{(*)}
\ecvlanguage{English}{\ecvBTwo}{\ecvCOne}{\ecvBTwo }{\ecvBTwo }{\ecvBTwo }
\ecvlanguagefooter[10pt]{(*)}


\ecvitem{\large Actual Tech stack and competences}{}
\ecvitem{ }{$\bigcdot$ \textbf{Operating System Experiences}}
\ecvitem{ }{ Linux (debian, ubuntu, centos)}

\ecvitem{ }{$\bigcdot$ \textbf{Programming Languages}}
\ecvitem{ }{ C++ 11+, Python 3, SQL,  PL/pgSQL, Java / Scala(basic level)}

\ecvitem{ }{$\bigcdot$ \textbf{Database Management Systems}}
\ecvitem{ }{ PostgreSQL, Oracle, Redis, InfluxDB, Cassandra, ClickHouse (basic level)}


\ecvitem{ }{$\bigcdot$ \textbf{Python technologies \& frameworks}}
\ecvitem{ }{ Tornado, aiohttp, sqlalchemy, WAMP}

\ecvitem{ }{$\bigcdot$ \textbf{C++ technologies \& frameworks}}
\ecvitem{ }{ cmake, STL, libevent, nghhtp2}


\ecvitem{ }{$\bigcdot$ \textbf{Data Streaming \& Services Bus}}
\ecvitem{ }{ RabbitMQ, Apache Kafka, Temporal.io }

\ecvitem{ }{$\bigcdot$ \textbf{Orchestration \& Containerization}}
\ecvitem{ }{ k8s, docker compose, docker swarm }

\ecvitem{ }{$\bigcdot$ \textbf{Cloud Computing Services}}
\ecvitem{ }{ GCP k8s, Google Pub/Sub}

\ecvitem{ }{$\bigcdot$ \textbf{Infra \& CI tools}}
\ecvitem{ }{ Jenkins CI, Gitlab CI, Ansible, Terraform}

\ecvitem{ }{$\bigcdot$ \textbf{Development tools}}
\ecvitem{ }{ PyCharm, CLion, Intellij Idea}

\ecvitem{ }{$\bigcdot$ \textbf{Version Control Systems}}
\ecvitem{ }{ Git}

\ecvitem{ }{$\bigcdot$ \textbf{Other skills}}
\ecvitem{ }{ Mathematics: MatLab, R Studio}
\ecvitem{ }{ Simulation: Rational Rose, Anylogic}

\ecvitem[10pt]{ }{ }


\ecvitem{SOURCE CODE}{}
\ecvitem{ }{\textbf{Source code, demonstration video and documentation of my projects: } }
\ecvitem{ }{\url{https://github.com/rshafeev}}

\ecvsection{Education and training}

\ecvitem{Place and Date}{National Technical University "Kharkov Polytechnic Institute", Ukraine, 2013 -- 2016}
\ecvitem{Specialty}{Mathematical modeling and computational methods}
\ecvitem{Title of qualification awarded}{passed PhD minimum, successful completion of postgraduate study}
\ecvitem[15pt]{Thesis theme}{Development of mathematical models and methods to solve the Dynamic Vehicle Routing Problem with uncertain input parameters}

\ecvitem{Place and Date}{National Technical University "Kharkov Polytechnic Institute", Ukraine}
\ecvitem{ }{Computer Mathematics and Mathematical Modeling department, 2011 -- 2013}
\ecvitem[15pt]{Title of qualification awarded}{Master`s degree in Applied Mathematics with excellence}

\ecvitem{Place and Date}{National Technical University "Kharkov Polytechnic Institute", Ukraine}
\ecvitem{ }{Computer Mathematics and Mathematical Modeling department, 2007 -- 2011}
\ecvitem{Title of qualification awarded}{Bachelor`s degree in Applied Mathematics with excellence}
\ecvitem{Principal subjects covered}{Mathematical Analysis}
\ecvitem{ }{Discrete Mathematics}
\ecvitem{ }{Programming (C,C++)}
\ecvitem{ }{Probability Theory and Mathematical Statistics}
\ecvitem{ }{Object Oriented Programming}
\ecvitem{ }{Numerical Methods}
\ecvitem{ }{Optimization Methods}
\ecvitem{ }{Logical Algorithms and Artificial Intelligence Systems}
\ecvitem{ }{Control Theory}
\ecvitem{ }{Development of Information Systems (Java, IDEF, Web 2.0)}
\ecvitem{ }{Computer Simulation}
\ecvitem{ }{Distributed Information Systems(Oracle)}
\ecvitem{ }{Actuarial Mathematics}



\ecvitem{PUBLICATIONS}{
	\begin{list}{\textbullet}{}
	  \item R. Shafeev.  Investigation of tuning parameters of Tabu Search algorithm and its modification  for solving the static Routing Courier Delivery Problem.  Kharkov NTU "KhPI" , 2016, 18 p.
	  
	 \item Lyubchyk L.M., Kolbasin V.A., Shafeev R.A. Nonlinear Signal Reconstruction based on Recursive Moving Window Kernel Method. / IDAACS, Warsaw, Poland, 2015, 6 p.
	  	  
	  \item R. Shafeev. A new metaheuristic algorithm for Solving the Transportation Problem with Time Constraints / L. Lyubchik // Vestnik NTU "KhPI". -- Kharkov: NTU "KhPI", 2013.  -- No3 (977). -- p. 35--39.   
	  
	 \item Shafeev R.A. A Development of SaaS service to solve dynamic vehicle routing problem / System analysis and information technologies: SAIT, Kyiv, 2013.
	  	  
	  \item R. Shafeev. Relationship between the Vehicle Routing Problem with Time Windows and the Assignment Problem.  // Theoretical and Applied Aspects of Cybernetics.  -- Kiev: Bukrek, 2012. -- p.145--149.
	  
	\end{list}
}
\ecvitem[5pt]{ }{ }
		
\ecvitem{SCIENTIFIC WORK}{
\begin{list}{\textbullet}{}

	\item May 2013, I presented the research work, devoted of development of client-server information system for solving the Dynamic Vehicle Routing Problem at the XV International Conference on Science and Technology "System	Analysis and Information Technologies" at the National Technical University “KPI”, Kiev, Ukraine.
	
	\item March 2012, The winner (1`st place) of the all-Ukrainian competition of the research student works, section "Informatics and Cybernetics", Vinnytsia, Ukraine.
	
	\item September 2011, participant of the International Conference of Logistics at the Hamburg	University of Technology, Hamburg, Germany.
	
	\item October 2010, I presented the research work, devoted to effects of electromagnetic fields on the complex biological objects at the Vth International conference “Environmental aspects of the technological security of the regions“ at the National Automobile and Road University, Kharkov, Ukraine.
	
	\item May 2010, I presented the research work, devoted to numerical simulation of the motion of celestial bodies at the XII International Conference on Science and Technology “System	Analysis and Information Technologies“ at the National Technical University "KPI", Kiev, Ukraine.
	
	\item May 2007, The winner (2nd place) of the third stage of the all-Ukrainian competition of research carried out by the students-members of the Ukrainian Small Academy of Sciences, section "Computer networks, databases and data banks", Kiev, Ukraine.
	
	\item December 2006, The winner (1nd place) of the second stage of the all-Ukrainian competition of research carried out by the students-members of the Ukrainian Small Academy of Sciences, section "Computer networks, databases and data banks", Zaporozhye, Ukraine.	
	
\end{list}
}



\ecvsection{Additional information}
\ecvitem{GRANTS}{Grant of Government of Ukraine, 2010--2011.}
\ecvitem[10pt]{}{Grant of the “DAAD-East European Partnership Exchange” funding framework between “National	Technical University” (Kharkov, Ukraine) and “Hamburg University of Technology-TUHH” (Germany). During	the internship, I worked as a team member, which developed Supply Chain Management project, Hamburg (Germany), July -- October 2011.}


\end{europecv}


\end{document} 