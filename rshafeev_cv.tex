%flagCMYK
\documentclass[helvetica,openbib,nologo,notitle,totpages]{europecv}
\usepackage[T1]{fontenc}
\usepackage{graphicx}
\usepackage[a4paper,top=1cm,left=1cm,right=1cm,bottom=2cm]{geometry}
\usepackage[english]{babel}
\usepackage{url}
\usepackage{hyperref}
\usepackage{xcolor}

\makeatletter
\newcommand*\bigcdot{\mathpalette\bigcdot@{.5}}
\newcommand*\bigcdot@[2]{\mathbin{\vcenter{\hbox{\scalebox{#2}{$\m@th#1\bullet$}}}}}
\makeatother

\ecvname{Shafeev, Roman}
\ecvfootername{Shafeev Roman}
\ecvaddress{Saint-Petersburg, Russia}

\ecvemail{\url{r.a.shafeev@yandex.com}, telegram:  @rshafeev}


\ecvnationality{Russian}
\ecvdateofbirth{Feb 14 1990}
\ecvgender{male}
\ecvpicture[width=4cm]{photo3.jpg}
\ecvfootnote{For more information call me}

\begin{document}
\selectlanguage{english}


\begin{europecv}
\ecvpersonalinfo[5pt]

\ecvsection{Work experience}

% ========== BUREAU 1440 ========= %

\ecvitem{}{}
\ecvitem{\textbf{Date}}{\textbf{January 2024 -- November 2024  \textcolor{gray}{ 11 months} }} 
\ecvitem{\textbf{Company}}{\textbf{BUREAU 1440}}
\ecvitem{\textbf{Position Held}}{\textbf{ Head of Architectural Design Department }}
\ecvitem{Roles}{
    $\bigcdot$ Leading a group of architects and system analysts
    
	$\bigcdot$ Architecture Design
	
	$\bigcdot$ Solution Prototypes Building
	
	$\bigcdot$ Iterative architecture design of Satellite Internet Constellation Network Management System
}

\ecvitem{Key Achievements}{
	$\bigcdot$ implemented SDLC for architects and system analysts.
	
	$\bigcdot$ implemented the Architecture as Code approach and organized the development process using ADRs.
	
	$\bigcdot$  made an HLD for a Unified Space Communications Network Management System based on two previously developed systems: network management system of a ground segment communications and a satellite flight management system.  The purpose of this unification is to eliminate duplication of software components and to build a solution for controlling satellite`s devices, ground and gateway stations of the space system on all available communication lines using unified principles. The unified management system allows to remote configuration, to execute control actions (for example, impulses for satellite maneuvering), to update satellite software, to collect telemetry, metrics, logs, and to detect faults in automatic mode to promptly recover the operability of the network element or satellite.
	
	$\bigcdot$ designed a new Earth-Satellite communication protocol for a management system for the purpose of managing and monitoring the state of satellite devices in an IPv6 network under conditions of high data losses at the physical level.
	
	$\bigcdot$ designed a workflow orchestration subsystem that helps operational operators coordinate work (configuration, software updates, maneuvering, etc.) for their mutual consistency in a semi-automatic mode.
	
	$\bigcdot$ designed a solution for automating the process of refining and updating the forecast of the spacecraft's trajectory without the participation of the operation operator. 
	
	$\bigcdot$ organized the migration of the management system to k8s.
		
}

% ========== MTS AI ========= %

\ecvitem{}{}
\ecvitem{\textbf{Date}}{\textbf{June 2022 -- December 2023  \textcolor{gray}{ 1 year 7 months } }} 
\ecvitem{\textbf{Company}}{\textbf{MTS AI}}
\ecvitem{\textbf{Position Held}}{\textbf{Chief Solution Architect }}
\ecvitem{Roles}{	
	$\bigcdot$ Architecture Design
	
	$\bigcdot$ Solution Prototypes Building

	$\bigcdot$ Iterative architecture design of the \href{https://mts.ai/ru/product/audiogram/}{Audiogram product} for speech synthesis and recognition (TTS/ASR) and the \href{https://mts.ai/ru/product/nlp-platform/}{Natural Language Processing Platform} to build intelligent virtual assistants.	
}

\ecvitem{Key Achievements}{

	$\bigcdot$ designed a solution for collecting and processing dialogues of communication between chatbots and users in the NLP Analytics system
	
	$\bigcdot$ designed Identity Access Management with an authorization model based on json-policies
	
	$\bigcdot$ designed a solution for identifying and verifying a speaker by his voice
	
	$\bigcdot$ designed a solution for determining the age group of a speaker during speech recognition from an audio stream
	
	$\bigcdot$ designed an Audio Archive for recording streaming audio for further use in "Speech Analytics" and for further training of ML models
	
	$\bigcdot$ designed a solution for horizontal scaling of Audiogram product services; organized the migration of the product to k8s
	
	$\bigcdot$ completed the transition from SOA to MSA model in the Audiogram product
	
	$\bigcdot$ implemented such engineering practices as 19-apps factors, observability, structured logging, distributed tracing of software components of the Audiogram product
	
}

\ecvitem{Technical Skills}{
	$\bigcdot$ Development: python, go
	
	$\bigcdot$ Services Bus: Apache Kafka
	
	$\bigcdot$ Databases: Postgresql, ClickHouse, OpenSearch, Redis, Milvus
	
	$\bigcdot$ Machine Learning:Triton Server, Rasa
	
}

% ========== RingCentral Analytics ========== %

\ecvitem{}{}
\ecvitem{\textbf{Date}}{\textbf{February 2021 -- January 2022 \textcolor{gray}{ / 1 year} }} 
\ecvitem{\textbf{Company}}{\textbf{RingCentral}}
\ecvitem{\textbf{Position Held}}{\textbf{Solution Architect }}
\ecvitem{Roles}{	
	$\bigcdot$ Architecture Design
	
	$\bigcdot$ Solution Prototypes Building
	
	$\bigcdot$ Iterative architecture design of the RingCentral Analytics Product
	
}

\ecvitem{Key Achievements}{
	$\bigcdot$ designed a solution to build "Quality as a Service" customer analytics for webinars with over 10K+ online attendees. The Analytics System uses video conference telemetry from RingCentral data points and from client applications to build quality score to determine the cause of failure and/or poor quality of media streaming for video conference participants.
	
	$\bigcdot$ designed technical solution to build `Webinar Marketing` Analytics. The Analytics System collects any actions of the attendee starting from the moment of registration in the webinar and ending with the completion of his participation in the webinar to build funnel trends for Marketers.
	
	$\bigcdot$ designed technical solution to async generate and download Analytics Reports from RingCentral Analytics Portal.
	
	$\bigcdot$ designed technical solution to reorganize RingCentral Analytics System to ensure GDPR and Schrems II Compliance and built regional clusters for EU and APAC.

	$\bigcdot$ made an audit of architectural documentation and actualized system and functional specifications of RingCentral Analytics System; built py-tool which generate draw.io interactive architectural diagrams from configuration files and applications artifacts and publish them into Confluence using Confluence API and integrated him to GitLab CI. 
}

\ecvitem{Technical Skills}{
	$\bigcdot$ Development: scala, sbt
	
	$\bigcdot$ Cloud Computing Services: GCP k8s, GCP Pub/Sub, Segment	
	
	$\bigcdot$ Services Bus: Apache Kafka, Temporal.io
	
	$\bigcdot$ Databases: Cassandra, ClickHouse, rocksdb, lucene
	
}

% ========== RingCentral Telco ========== %

\ecvitem{}{}
\ecvitem{\textbf{Date}}{\textbf{Sep 2019 -- February 2021 \textcolor{gray}{ / 1 year 6 months} }} 
\ecvitem{\textbf{Company}}{\textbf{RingCentral}}
\ecvitem{\textbf{Position Held}}{\textbf{Team Lead ( Telephony - Infra ) }}
\ecvitem{Roles}{	
	$\bigcdot$ Team Leading \& Scrum Master
	
	$\bigcdot$ Projects Roadmap Planning	
	
	$\bigcdot$ Architecture Design
	
	$\bigcdot$ C++ Developing
}

\ecvitem{Key Achievements}{
	$\bigcdot$ built Development Process based on Self Agile methodology and increased responsibility for team actions,  outcomes and delivers on commitments. 
	
	$\bigcdot$ migrated legacy Kernel Telephony Service (millions code lines) which has a lot of win-specific code (winapi, com objects, mfc, etc.) from Win to Linux Platform. 
	
	$\bigcdot$ refactored legacy Kernel Telephony Service using EDA architecture concepts to simplify the internal architecture of the service and migrate to an asynchronous model of communication between isolated service components.
	
}

\ecvitem{Technical Skills}{
	$\bigcdot$ Development: c++11, cmake, boost, stl, protobuf
	
	$\bigcdot$ Orchestration: systemd	
	
	$\bigcdot$ Services Bus: Apache Kafka
	
	$\bigcdot$ Databases: Oracle
	
}

% ========== Arrival Leading ========== %

\ecvitem{}{}
\ecvitem{\textbf{Date}}{\textbf{March 2018 --July 2019 \textcolor{gray}{ / 1 year 5 months} }} 
\ecvitem{\textbf{Company}}{\textbf{ARRIVAL LTD}}
\ecvitem{\textbf{Position Held}}{\textbf{Technical Lead ( Connected Vehicle Cloud ) }}
\ecvitem{Roles}{	
	$\bigcdot$ Team Leading \& Scrum Master
	
	$\bigcdot$ Product Roadmap Planning	

	$\bigcdot$ Architecture Design
}

\ecvitem{Key Achievements}{
	$\bigcdot$ formed a product team with 12 members
	
	$\bigcdot$ built the Connected Vehicle Cloud for Arrival vehicles, allowing to collect, process and visualize any vehicle sensor data from CAN/ETH buses in real time
	
	$\bigcdot$ released a Remote Diagnostics product that allows service engineers to quickly determine the cause of software/hardware failures in an electric vehicle when they only have remote access to the vehicle
	
	$\bigcdot$ released the OTA-Update product for remote updating of applications and firmware of automotive IO modules, HMI and automotive systems based on Linux  	
	
	$\bigcdot$ designed and implemented a new automotive protocol for $vehicle \leftrightarrow cloud$ communications. The protocol has the following properties: multi-streaming per single tcp connection, streams prioritization, datagrams compression, data tls encryption, guarantee delivery, saving traffic when transmitting telemetry (only telemetry signal changes are transmitted). 
}

\ecvitem{Technical Skills}{
	$\bigcdot$ Connected Vehicle Development: c++11, cmake, nghhtp2, stl, protobuf
	
	$\bigcdot$ API: python3, tornado, sqlalchemy, WAMP
	
	$\bigcdot$ Orchestration: k8s+terraform+helm, docker compose	
	
	$\bigcdot$ Cloud Computing Services: GCP k8s
	
	$\bigcdot$ Services Bus: RabbitMQ Cluster
	
	$\bigcdot$ Databases: PostgreSQL, InfluxDB
	
}

% ========== Starline Leading ========== %
\ecvitem{}{}
\ecvitem{\textbf{Date}}{\textbf{February 2017 -- March 2018 \textcolor{gray}{ / 1 year 2 months} }} 
\ecvitem{\textbf{Company}}{\textbf{StarLine}}
\ecvitem{\textbf{Position Held}}{\textbf{Technical Lead ( Insurance Telematics) }}
\ecvitem{Roles}{	
	$\bigcdot$ Team Leading \& Scrum Master
	
	$\bigcdot$ Product Roadmap Planning
	
    $\bigcdot$ Technical Integration \& Communication with Telematics Partners
	
	$\bigcdot$ Architecture Design

	$\bigcdot$ Python API Backend Developing		
}

\ecvitem{Key Achievements}{
	$\bigcdot$ designed \& developed Pub/Sub API for b2b integration with Telematics Providers and Insurance Partners to transfer vehicles telemetry to their systems in real-time mode.

	$\bigcdot$ implemented 'the driving style assessment' feature to calculate driving style score by a selected date-time period in real-time mode.
	
	$\bigcdot$  implemented the 'insurance telematics' b2b product to calculate insurance score and crash detection/reconstruction based on raw data of vehicle accelerometer which collected from the vehicle on the cloud side.		
}

\ecvitem{Technical Skills}{

	$\bigcdot$ Development: python3, tornado, sqlalchemy
	
}

% ========== Starline Developing ========== %
\ecvitem{}{}
\ecvitem{\textbf{Date}}{\textbf{July 2015 -- February 2017 \textcolor{gray}{ / 1 year 8 months} }} 
\ecvitem{\textbf{Company}}{\textbf{StarLine}}
\ecvitem{\textbf{Position Held}}{\textbf{C++/Python Backend Developer}}
\ecvitem{Roles}{	
$\bigcdot$ C++ Backend Developing

$\bigcdot$ Python API Backend Developing

$\bigcdot$ Technical Design

$\bigcdot$ Applied Mathematics Research

}

\ecvitem{Key Achievements}{
	$\bigcdot$ designed and implemented  MSA Architectural concepts for Telemetry Monoliths Server Decomposition using RabbitMQ as Services Messages Bus.
	
	$\bigcdot$ designed and implemented remote communication between client applications(mobile apps, web portal) and connected vehicles through $client \leftrightarrow api \leftrightarrow platform \leftrightarrow vehicle$ on Connected Vehicle Platform side using redis pub/sub channels.
	
	$\bigcdot$ developed backend micro-services with horizontal scaling to receive and to collect vehicle`s telemetry of more than 300K connected vehicles.
	
	$\bigcdot$ designed and implemented a heuristic algorithm to solve 'anti-star' problem to filter noise and broken track points in real-time.
}

\ecvitem{Technical Skills}{
$\bigcdot$ Connected Vehicle Development: c++11, cmake, poco, stl, libevent, protobuf

$\bigcdot$ API Development: python3, tornado, sqlalchemy

$\bigcdot$ Services Communication Bus: RabbitMQ Cluster (via SimpleAmqpClient for c++ / via pika for python)

$\bigcdot$ Databases:  redis (via hiredis), mysql (via mysqlcppconn), oracle (via occi)
 
}

% ========== KhPI Developing ========== %
\ecvitem{}{}
\ecvitem{}{}
\ecvitem{\textbf{Date}}{\textbf{September 2013 -- June 2015 \textcolor{gray}{ / 1 year 10 months} }} 
\ecvitem{\textbf{Company}}{\textbf{NTU "KhPI", Ukraine}}
\ecvitem{\textbf{Position Held}}{\textbf{Researcher of the Department of Computer Mathematics and Mathematical Modeling}}
\ecvitem{Roles}{
		$\bigcdot$ Applied Mathematics Research
		
		$\bigcdot$ Conducting laboratory and practical classes of C++ programming
}
\ecvitem{Key Achievements}{
		$\bigcdot$ designed and implemented matlab application which allows finding the best productive supply for each transformer with minimal losses on the  power transformers in Dushanbe (Tajikistan).
		
		$\bigcdot$  developed a vehicle routing java framework that uses specialized metaheuristic algorithms to calculate an optimal solution of the different classes of the static and dynamic vehicle routing problems.
}
\ecvitem{Technical Skills}{
	$\bigcdot$ Language: C++, Java
	
	$\bigcdot$ Math Tools: Matlab
}


% ========== TUHH ========== %
\ecvitem{}{}
\ecvitem{}{}
\ecvitem{\textbf{Date}}{\textbf{July 2011 -- November 2011 \textcolor{gray}{ / 5 months} }} 
\ecvitem{\textbf{Company}}{\textbf{Hamburg University of Technology-TUHH}}
\ecvitem{\textbf{Position Held}}{\textbf{Software Developer}}
\ecvitem{Roles}{
	$\bigcdot$ Applied Mathematics Research
	
	$\bigcdot$ C++ / Java Developing
}
\ecvitem{Key Achievements}{
	$\bigcdot$ built model which follows a rigorous development process framework, where model	validity is ensured by using Supply Chain Operations Reference as theoretical process	framework using Anylogic Modeling Platform. An agent based simulation platform is presented for generic supply chain modeling adding flexibility and configurability over existing models.
	
	$\bigcdot$ developed UI-tool which allows to design delivery routes between the points of the delivery chains and export them  in .accdb format to have the ability to use them in Anylogic into supply chain model.
}

\ecvitem{Technical Skills}{
	
	$\bigcdot$ Modeling Tools: Anylogic 6.6 
	
	$\bigcdot$ Development: C++, WinAPI/MFC, Visual Studio 2008, OpenStreetMap
}



\ecvitem{}{}
\ecvitem{}{}
\ecvsection{Personal skills and~competences}

\ecvmothertongue[5pt]{Russian}
\ecvitem{\large Other language(s)}{English}
\ecvlanguageheader{(*)}
\ecvlanguage{English}{\ecvBTwo}{\ecvCOne}{\ecvBTwo }{\ecvBTwo }{\ecvBTwo }
\ecvlanguagefooter[10pt]{(*)}


\ecvitem{\large Actual Technical skills and competences}{}
\ecvitem{ }{$\bigcdot$ \textbf{Operating System Experiences}}
\ecvitem{ }{ Linux (debian, ubuntu, centos)}

\ecvitem{ }{$\bigcdot$ \textbf{Programming Languages}}
\ecvitem{ }{ C++ 11+, Python 3, SQL,  PL/pgSQL, Java / Scala(basic level)}

\ecvitem{ }{$\bigcdot$ \textbf{Database Management Systems}}
\ecvitem{ }{ PostgreSQL, Oracle, Redis, InfluxDB, Cassandra, ClickHouse (basic level)}


\ecvitem{ }{$\bigcdot$ \textbf{Python technologies \& frameworks}}
\ecvitem{ }{ Tornado, aiohttp, sqlalchemy, WAMP}

\ecvitem{ }{$\bigcdot$ \textbf{C++ technologies \& frameworks}}
\ecvitem{ }{ cmake, STL, libevent, nghhtp2}


\ecvitem{ }{$\bigcdot$ \textbf{Data Streaming \& Services Bus}}
\ecvitem{ }{ RabbitMQ, Apache Kafka, Temporal.io }

\ecvitem{ }{$\bigcdot$ \textbf{Orchestration \& Containerization}}
\ecvitem{ }{ k8s, docker compose, docker swarm }

\ecvitem{ }{$\bigcdot$ \textbf{Cloud Computing Services}}
\ecvitem{ }{ GCP k8s, Google Pub/Sub}

\ecvitem{ }{$\bigcdot$ \textbf{Infra \& CI tools}}
\ecvitem{ }{ Jenkins CI, Gitlab CI, Ansible, Terraform}

\ecvitem{ }{$\bigcdot$ \textbf{Development tools}}
\ecvitem{ }{ PyCharm, CLion, Intellij Idea}

\ecvitem{ }{$\bigcdot$ \textbf{Version Control Systems}}
\ecvitem{ }{ Git}

\ecvitem{ }{$\bigcdot$ \textbf{Other skills}}
\ecvitem{ }{ Mathematics: MatLab, R Studio}
\ecvitem{ }{ Simulation: Rational Rose, Anylogic}

\ecvitem[10pt]{ }{ }


\ecvitem{SOURCE CODE}{}
\ecvitem{ }{\textbf{Source code, demonstration video and documentation of my projects: } }
\ecvitem{ }{\url{https://github.com/rshafeev}}

\ecvsection{Education and training}

\ecvitem{Place and Date}{National Technical University "Kharkov Polytechnic Institute", Ukraine, 2013 -- 2016}
\ecvitem{Specialty}{Mathematical modeling and computational methods}
\ecvitem{Title of qualification awarded}{passed PhD minimum, successful completion of postgraduate study}
\ecvitem[15pt]{Thesis theme}{Development of mathematical models and methods to solve the Dynamic Vehicle Routing Problem with uncertain input parameters}

\ecvitem{Place and Date}{National Technical University "Kharkov Polytechnic Institute", Ukraine}
\ecvitem{ }{Computer Mathematics and Mathematical Modeling department, 2011 -- 2013}
\ecvitem[15pt]{Title of qualification awarded}{Master`s degree in Applied Mathematics with excellence}

\ecvitem{Place and Date}{National Technical University "Kharkov Polytechnic Institute", Ukraine}
\ecvitem{ }{Computer Mathematics and Mathematical Modeling department, 2007 -- 2011}
\ecvitem{Title of qualification awarded}{Bachelor`s degree in Applied Mathematics with excellence}
\ecvitem{Principal subjects covered}{Mathematical Analysis}
\ecvitem{ }{Discrete Mathematics}
\ecvitem{ }{Programming (C,C++)}
\ecvitem{ }{Probability Theory and Mathematical Statistics}
\ecvitem{ }{Object Oriented Programming}
\ecvitem{ }{Numerical Methods}
\ecvitem{ }{Optimization Methods}
\ecvitem{ }{Logical Algorithms and Artificial Intelligence Systems}
\ecvitem{ }{Control Theory}
\ecvitem{ }{Development of Information Systems (Java, IDEF, Web 2.0)}
\ecvitem{ }{Computer Simulation}
\ecvitem{ }{Distributed Information Systems(Oracle)}
\ecvitem{ }{Actuarial Mathematics}



\ecvitem{PUBLICATIONS}{
	\begin{list}{\textbullet}{}
	  \item R. Shafeev.  Investigation of tuning parameters of Tabu Search algorithm and its modification  for solving the static Routing Courier Delivery Problem.  Kharkov NTU "KhPI" , 2016, 18 p.
	  
	 \item Lyubchyk L.M., Kolbasin V.A., Shafeev R.A. Nonlinear Signal Reconstruction based on Recursive Moving Window Kernel Method. / IDAACS, Warsaw, Poland, 2015, 6 p.
	  	  
	  \item R. Shafeev. A new metaheuristic algorithm for Solving the Transportation Problem with Time Constraints / L. Lyubchik // Vestnik NTU "KhPI". -- Kharkov: NTU "KhPI", 2013.  -- No3 (977). -- p. 35--39.   
	  
	 \item Shafeev R.A. A Development of SaaS service to solve dynamic vehicle routing problem / System analysis and information technologies: SAIT, Kyiv, 2013.
	  	  
	  \item R. Shafeev. Relationship between the Vehicle Routing Problem with Time Windows and the Assignment Problem.  // Theoretical and Applied Aspects of Cybernetics.  -- Kiev: Bukrek, 2012. -- p.145--149.
	  
	\end{list}
}
\ecvitem[5pt]{ }{ }
		
\ecvitem{SCIENTIFIC WORK}{
\begin{list}{\textbullet}{}

	\item May 2013, I presented the research work, devoted of development of client-server information system for solving the Dynamic Vehicle Routing Problem at the XV International Conference on Science and Technology "System	Analysis and Information Technologies" at the National Technical University “KPI”, Kiev, Ukraine.
	
	\item March 2012, The winner (1`st place) of the all-Ukrainian competition of the research student works, section "Informatics and Cybernetics", Vinnytsia, Ukraine.
	
	\item September 2011, participant of the International Conference of Logistics at the Hamburg	University of Technology, Hamburg, Germany.
	
	\item October 2010, I presented the research work, devoted to effects of electromagnetic fields on the complex biological objects at the Vth International conference “Environmental aspects of the technological security of the regions“ at the National Automobile and Road University, Kharkov, Ukraine.
	
	\item May 2010, I presented the research work, devoted to numerical simulation of the motion of celestial bodies at the XII International Conference on Science and Technology “System	Analysis and Information Technologies“ at the National Technical University "KPI", Kiev, Ukraine.
	
	\item May 2007, The winner (2nd place) of the third stage of the all-Ukrainian competition of research carried out by the students-members of the Ukrainian Small Academy of Sciences, section "Computer networks, databases and data banks", Kiev, Ukraine.
	
	\item December 2006, The winner (1nd place) of the second stage of the all-Ukrainian competition of research carried out by the students-members of the Ukrainian Small Academy of Sciences, section "Computer networks, databases and data banks", Zaporozhye, Ukraine.	
	
\end{list}
}



\ecvsection{Additional information}
\ecvitem{GRANTS}{Grant of Government of Ukraine, 2010--2011.}
\ecvitem[10pt]{}{Grant of the “DAAD-East European Partnership Exchange” funding framework between “National	Technical University” (Kharkov, Ukraine) and “Hamburg University of Technology-TUHH” (Germany). During	the internship, I worked as a team member, which developed Supply Chain Management project, Hamburg (Germany), July -- October 2011.}


\end{europecv}


\end{document} 