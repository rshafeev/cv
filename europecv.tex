%flagCMYK
\documentclass[helvetica,openbib,nologo,notitle,totpages]{europecv}
\usepackage[T1]{fontenc}
\usepackage{graphicx}
\usepackage[a4paper,top=1cm,left=1cm,right=1cm,bottom=2cm]{geometry}
\usepackage[english]{babel}
\usepackage{url}


\ecvname{Shafeyev, Roman}
\ecvfootername{Shafeyev Roman}
\ecvaddress{32 f.6 Krichevsky St, 61027, Kharkov, Ukraine}
\ecvtelephone[+38 097 940 39 92]{+38 057 367 94 93	 Skype ID: roma.shafeyev }
\ecvemail{\url{roman.shafeyev@gmail.com}, \url{rs@introunion.com}}


\ecvnationality{Russian}
\ecvdateofbirth{Feb 14 1990}
\ecvgender{male}
\ecvpicture[width=4cm]{photo.png}
\ecvfootnote{For more information call me}

\begin{document}
\selectlanguage{english}


\begin{europecv}
\ecvpersonalinfo[5pt]

\ecvsection{Work experience}

\ecvitem{Date}{September 2013 -- Now}
\ecvitem{Occupation or position held}{Lecturer at the Department of Computer Mathematics and Mathematical Modeling, NTU "KhPI"}


\ecvsection{Education and training}

\ecvitem{Place and Date}{National Technical University "Kharkov Polytechnic Institute", Ukraine, 2013 -- Now}
\ecvitem{Specialty}{Mathematical modeling and computational methods}
\ecvitem{Title of qualification awarded}{Candidate of Engineering Sciences}
\ecvitem[15pt]{Thesis theme}{Development of mathematical models and methods to solve the Dynamic Vehicle Routing Problem with uncertain input parameters}

\ecvitem{Place and Date}{National Technical University "Kharkov Polytechnic Institute", Ukraine}
\ecvitem{ }{Computer Mathematics and Mathematical Modeling department, 2011 -- 2013}
\ecvitem[15pt]{Title of qualification awarded}{Master`s degree in Applied Mathematics with excellence}

\ecvitem{Place and Date}{National Technical University "Kharkov Polytechnic Institute", Ukraine}
\ecvitem{ }{Computer Mathematics and Mathematical Modeling department, 2007 -- 2011}
\ecvitem{Title of qualification awarded}{Bachelor`s degree in Applied Mathematics with excellence}
\ecvitem{Principal subjects covered}{Mathematical Analysis}
\ecvitem{ }{Discrete Mathematics}
\ecvitem{ }{Programming (C,C++)}
\ecvitem{ }{Probability Theory and Mathematical Statistics}
\ecvitem{ }{Object Oriented Programming}
\ecvitem{ }{Numerical Methods}
\ecvitem{ }{Optimization Methods}
\ecvitem{ }{Logical Algorithms and Artificial Intelligence Systems}
\ecvitem{ }{Control Theory}
\ecvitem{ }{Development of Information Systems (Java, IDEF, Web 2.0)}
\ecvitem{ }{Computer Simulation}
\ecvitem{ }{Distributed Information Systems(Oracle)}
\ecvitem{ }{Actuarial Mathematics}


\ecvsection{Personal skills and~competences}

\ecvmothertongue[5pt]{Russian}
\ecvitem{\large Other language(s)}{English}
\ecvlanguageheader{(*)}
\ecvlanguage{English}{\ecvBOne}{\ecvCOne}{\ecvBTwo }{\ecvBTwo }{\ecvBTwo }
\ecvlanguagefooter[10pt]{(*)}


\ecvitem{\large Computer skills and competences}{}
\ecvitem{ }{\textbf{Operating System Experiences}}
\ecvitem{ }{ - Linux (Ubuntu /Ubuntu Server), MS Windows}

\ecvitem{ }{\textbf{Programming Languages}}
\ecvitem{ }{ - C/C++, C\#, Java, Python(writing scripts), Ruby, SQL,  PL/pgSQL}

\ecvitem{ }{\textbf{Web-based technologies}}
\ecvitem{ }{ - HTML, JavaScript, EmberJS,  ASP.NET MVC,  Spring MVC, JSP/FreeMarker, Hibernate, Ruby on Rails 4}

\ecvitem{ }{\textbf{Continuous integration tools}}
\ecvitem{ }{ - Maven, Jenkins CI, GruntJS, Capistrano}

\ecvitem{ }{\textbf{Development tools}}
\ecvitem{ }{ - Intellij Idea, Eclipse, RubyMine, QT Creator,  MS Visual Studio}

\ecvitem{ }{\textbf{Database Management Systems}}
\ecvitem{ }{ - MS SQL Server, PostgreSQL, SQLite, Oracle}

\ecvitem{ }{\textbf{Version Control Systems}}
\ecvitem{ }{ - Git, SVN}

\ecvitem{ }{\textbf{Other skills}}
\ecvitem{ }{ - Mathematics: MatLab, Mathcad, R Studio}
\ecvitem{ }{ - Simulation: Rational Rose, GPSSW, Anylogic}
\ecvitem{ }{ - Graphics: OpenGL(+shaders)}

\ecvitem[10pt]{ }{ }

\ecvsection{Additional information}
\ecvitem{GRANTS}{Grant of Government of Ukraine, 2010--2011.}
\ecvitem[10pt]{}{Grant of the “DAAD-East European Partnership Exchange” funding framework between “National	Technical University” (Kharkov, Ukraine) and “Hamburg University of Technology-TUHH” (Germany). During	the internship, I worked as a team member, which developed Supply Chain Management project, Hamburg (Germany), July -- October 2011.}

\ecvitem{PROJECT EXPERIENCE}{}
\ecvitem{ }{\textbf{OptSDK -- Java-based framework for evolutionary computation. } }
\ecvitem{ }{(September 2014 -- Now)}
\ecvitem{ }{\textit{Used technologies and tools}}
\ecvitem{ }{- Basic: Java, Intellij Idea}
\ecvitem{ }{\textit{Description}}
\ecvitem{ }{The goal of OptSDK is to simplify the evolutionary optimization of user-defined problems as well as the implementation of arbitrary metaheuristic optimization algorithms.}

\ecvitem{ }{\textbf{JLogistics -- Vehicle Routing software framework} (December 2013 -- Now)}
\ecvitem{ }{\textit{Used technologies and tools}}
\ecvitem{ }{- Basic:  Java, Intellij Idea}
\ecvitem{ }{- Web:  Spring MVC, JSP/FreeMarker}
\ecvitem{ }{\textit{Description}}
\ecvitem{ }{JLogistics is a vehicle routing software framework for Java that uses specialized metaheuristic algorithms to calculate an optimal solution of the different classes of the static and dynamic vehicle routing problems.}
\ecvitem[5pt]{ }{ }

\ecvitem{ }{\textbf{An application for computing the optimal productive supply of the power transformers in Dushanbe (Tajikistan).}(September 2012 -- March 2013)}
\ecvitem{ }{\textit{Used technologies and tools}}
\ecvitem{ }{- Basic:  Matlab}
\ecvitem{ }{\textit{Description}}
\ecvitem{ }{The developed application allows to find the best productive supply for each transformer with minimal losses on the transformers.}
\ecvitem[5pt]{ }{ }


\ecvitem{ }{\textbf{Supply Chain Building Blocks} (July 2011 -- February 2012)}
\ecvitem{ }{\textit{Used technologies and tools}}
\ecvitem{ }{- Basic:  Java, Anylogic 6.6}
\ecvitem{ }{- Routes building for transport agents:   C++, WinAPI/MFC, Visual Studio 2008, OSM}
\ecvitem{ }{- Database: Microsoft Excel (with macros), Microsoft Access}
\ecvitem{ }{\textit{Description}}
\ecvitem{ }{The modeling platform follows a rigorous development process framework, where model	validity is ensured by using Supply Chain Operations Reference as theoretical process	framework. An agent based simulation platform is presented for generic supply chain modeling adding flexibility and configurability over existing models.}
\ecvitem[5pt]{ }{ }

\ecvitem{ }{\textbf{Numerical simulation of the motion of celestial bodies}(October 2009--July 2010)}
\ecvitem{ }{\textit{Used technologies and tools}}
\ecvitem{ }{- Basic: C++, WinAPI/MFC, Visual Studio 2008}
\ecvitem{ }{- Database: MS SQL Server 2008 Express}
\ecvitem{ }{- Graphics: OpenGL}
\ecvitem{ }{\textit{Description}}
\ecvitem{ }{The scientific software is for the numerical decision of the research problem of celestial bodies movement processes, visualization in three-dimensional space of modeling process, as well as processing, ordering and classification of the received orbital data. The main project objective was to define potentially dangerous for the Earth asteroids from the Aton’s group and create the catalog of orbital evolution for them on a time interval from 2009 to 2200.}
\ecvitem[5pt]{ }{ }

\ecvitem{ }{\textbf{Navigation GIS} (October 2009 -- July 2010)}
\ecvitem{ }{\textit{Used technologies and tools}}
\ecvitem{ }{- Basic: C++, WinAPI/MFC, Visual Studio 2008}
\ecvitem{ }{- Database: PostgreSQL}
\ecvitem{ }{- Network: TCP/IP sockets}
\ecvitem{ }{- Render map: OpenStreetMap, Google Maps API}
\ecvitem{ }{\textit{Description}}
\ecvitem{ }{The client-server system was used for vehicle movement monitoring in real time (Student project).}
\ecvitem[5pt]{ }{ }

\ecvitem{ }{\textbf{Terrain Generator} (September 2008 -- January 2009)}
\ecvitem{ }{\textit{Used technologies and tools}}
\ecvitem{ }{- Basic: C++, WinAPI/MFC, Visual Studio 2008}
\ecvitem{ }{- Graphics: OpenGL, shaders}
\ecvitem{ }{\textit{Description}}
\ecvitem{ }{This application is the generator of three-dimensional landscapes (Student project).}
\ecvitem[5pt]{ }{ }

\ecvitem{PUBLICATIONS}{
	\begin{list}{\textbullet}{}
	  \item R. Shafeyev.  Investigation of tuning parameters of Tabu Search algorithm and its modification  for solving the static Routing Courier Delivery Problem.  // Theoretical and Applied Aspects of Cybernetics.  -- Kiev: Bukrek, 2015 (not yet published).
	  
	  \item R. Shafeyev. A new metaheuristic algorithm for Solving the Transportation Problem with Time Constraints / L. Lyubchik // Vestnik NTU "KhPI". -- Kharkov: NTU "KhPI", 2013.  -- No3 (977). -- p. 35--39.   
	  
	  \item R. Shafeyev. Relationship between the Vehicle Routing Problem with Time Windows and the Assignment Problem.  // Theoretical and Applied Aspects of Cybernetics.  -- Kiev: Bukrek, 2012. -- p.145--149.
	  
	\end{list}
}
\ecvitem[5pt]{ }{ }
		
\ecvitem{SCIENTIFIC WORK}{
\begin{list}{\textbullet}{}
	
	\item May 2013, I presented the research work, devoted of development of client-server information system for solving the Dynamic Vehicle Routing Problem at the XV International Conference on Science and Technology "System	Analysis and Information Technologies" at the National Technical University “KPI”, Kiev, Ukraine.
	
	\item March 2012, The winner (1`st place) of the all-Ukrainian competition of the research student works, section "Informatics and Cybernetics", Vinnytsia, Ukraine.
	
	\item September 2011, participant of the International Conference of Logistics at the Hamburg	University of Technology, Hamburg, Germany.
	
	\item October 2010, I presented the research work, devoted to effects of electromagnetic fields on the complex biological objects at the Vth International conference “Environmental aspects of the technological security of the regions“ at the National Automobile and Road University, Kharkov, Ukraine.
	
	\item May 2010, I presented the research work, devoted to numerical simulation of the motion of celestial bodies at the XII International Conference on Science and Technology “System	Analysis and Information Technologies“ at the National Technical University "KPI", Kiev, Ukraine.
	
	\item May 2007, The winner (2nd place) of the third stage of the all-Ukrainian competition of research carried out by the students-members of the Ukrainian Small Academy of Sciences, section "Computer networks, databases and data banks", Kiev, Ukraine.
	
	\item December 2006, The winner (1nd place) of the second stage of the all-Ukrainian competition of research carried out by the students-members of the Ukrainian Small Academy of Sciences, section "Computer networks, databases and data banks", Zaporozhye, Ukraine.	
	
\end{list}
}






\end{europecv}


\end{document} 